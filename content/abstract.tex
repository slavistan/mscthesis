\def\abstractWidth{14cm}

\NewDocumentCommand{\abstrTitleStyle}{ m }{\center{\Large\textbf{#1}}\\\vspace{5mm}}

\begin{tikzpicture}[
  remember picture,
  overlay
    ]
  \node[anchor=north,
        font=\normalsize,
        align=justify,
        text width=\abstractWidth] (ABSTRACT-EN)
    at ($(current page.north west)+(0.5\paperwidth, -5)$)
    {\abstrTitleStyle{Abstract}

      Many domains of the physical sciences such as computational fluid dynamics and electrodynamics use simulations
      based on known governing laws to gather insights into a system's transient or static behavior. Where geometries
      are regular structured grids are utilized in the modelling process for their good performance due to the inherent
      structural simplicity. The solution procedure involves matrix-vector multiplications on the sparse system matrix
      which are implemented using different general purpose sparse matrix storage formats, such as the compressed sparse
      row format (CSR). This thesis introduces the threefold compressed sparse row matrix storage format (C3SR) designed
      for compact storage and fast arithmetic of sparse banded matrices derived from structured grids. The storage
      format, its compression method and arithmetic schemes are presented and evaluated in a series of tests and
      benchmarks. Compression is found to perform very well and to be stable with respect to perturbations of a grid's
      regularity. Arithmetic performance is evaluated on three systems with varying, possibly promising results which
      suggest further inquiry. Follow-up work is suggested in the outlook.};

  \node[below=1cm of ABSTRACT-EN,
        align=justify,
        font=\normalsize,
        text width=\abstractWidth] (ABSTRACT-DE)
    {\abstrTitleStyle{Zusammenfassung}

    Viele Teilbereiche der Naturwissenschaft verwenden ihr Verständnis über die herrschenden Gesetze eines Systems um
    mittels Simulationen Einblicke in das Verhalten des Systems zu gewinnen. Probleme, deren Domäne aus regulären
    Strukturen aufgebaut ist werden häufig auf Grundlage von strukturierten Gittern modelliert um deren geringere
    Komplexität zur Beschleunigung der Simulation zu verwenden. Lösungsverfahren erzeugen eine dünnbesetze, aus
    digonalen Bändern aufgebaue Systemmatrix, welche für eine Vielzahl von Matrix-Vektor Multliplikationen benötigt
    wird, die mittels üblicher Speicherformate für dünnbesetze Matrizen implementiert wird. Diese Arbeit stellt eine
    Abwandlung des gebräuchlichen Compressed Sparse Row (CSR) Formats vor, das Threefold Compressed Sparse Row (C3SR)
    Format, welches für kompaktes Speichern und performante Arithmetik von dünnbesetzen Matrizen, welche von
    strukturierten Gittern abgeleitet wurden, entworfen wurde. Das Format, das Kompressionsmethode und die
    Multiplikationsschemata werden vorgestellt und auf Leistung untersucht. Das Kompressionsverfahren erzielt auch bei
    Störung der Regularität der Gitter sehr gute Ergebnisse. Die Matrix-Vektor Multiplikation wird auf 3 Systemen
    getestet und kann unterschiedliche, zum Teil aber erfolgsversprechende Ergebnisse aufweisen, welche weitere
    Analyse und Untersuchungen nahelegen. Vorschläge dafür sind im abschließenden Kapitel gelistet.
    };
\end{tikzpicture}

\newpage
