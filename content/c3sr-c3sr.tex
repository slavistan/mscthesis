\chapter{3-Fold Compressed Sparse Row Matrix Storage Format}

  Evidently, representing a sparse banded matrix using the regular CSR format is suboptimal, as it has no means of
  capturing the apparent repetitiveness of the structure. While at first glance the diagonal format might seem an
  appealing choice for the types of matrices introduced in the previous section, real-life problems produce matrices
  which are possibly only locally structured, i.e. they contain multiple fully structured sections corresponding to the
  multiple structured grid regions of the overall heterogeneous domain (Figure \ref{fig:refined_structured_grid}), which
  need not be aligned in a way to produce a diagonal structure at all. For such matrices the diagonal storage format is
  highly suboptimal \cite{Bell2011}. Thus a more flexible approach is taken adapting the general purpose CSR format to
  better suit the characteristics of sparse banded matrices.

  This section introduces the data layout and the matrix-vector multiplication scheme of the \keyword{3-fold compressed
  sparse row} matrix storage format (C3SR). Its raison d'être is to optimize arithmetic performance of matrix-vector
  multiplications involving sparse banded matrices by improving data locality through a structurally-aware space-saving
  storage scheme and by implementing an efficient arithmetic scheme capable of profiting from modern computer
  architectures' hardware assets.

  \section{Data Layout and Storage Scheme}

    It is crucial to observe that, in the most general case, a regularity inherent to a sparse banded matrix's
    non-zeros' column indices is not shared by the non-zeros' numerical values. While the non-zeros of two or more rows
    may share the same column indices, save for a possible offset, their corresponding values need not to be similar to
    each other at all. To prevent that a lack of common regularity impedes optimizing the storage of one or the other it
    is thus necessary to decouple the representation of a row's non-zeros' column index positions from the
    representation of their numerical values. The C3SR format accounts for this circumstance and maintains separate data
    structures for the non-zeros' column indices and their values. Hence the nomenclature of the C3SR format, as it is
    based on the idea of the CSR format and adds another two potential layers of compression: one for the column indices
    and one for the values.

    Based on the observation that sparse banded matrices contain many rows whose non-zeros are located at the same
    positions except for a possible offset the C3SR format decomposes the column index positions into the column index
    of the row's first non-zero element, referred to as the \keyword{row's peg} hereinafter, and the relative column
    indices of all of the row's non-zeros with respect to the first non-zero's column index, i.e. the relative offsets
    with respect to the peg index. The latter column index offsets relative to the peg index shall be called the
    \keyword{row's column index pattern}, or simply the \keyword{row's pattern}. Naturally, only unique patterns are stored,
    drastically shrinking the storage requirements of such matrices whose majority of rows exhibit the same pattern.

    Thus, in order to represent the column indices of a matrix row's non-zeros the C3SR format utilizes three arrays:

    \begin{description}[align = left, labelwidth = 4cm]
      \item [JP - \emph{Peg column indices}] \hfill \\
        Column index of each row's first non-zero. One element per row in matrix.
      \item [J - \emph{Column index patterns}] \hfill \\
        Column index position offsets of the rows' non-zeros relative to the peg
        column index for each row. Only unique patterns are stored.
      \item [JS - \emph{Patterns' index-pointers}] \hfill \\
        Index-pointer to each row's first element within J. One element per row in matrix.
    \end{description}

    An example matrix and its corresponding structural information are given in Figure \ref{fig:c3sr_example_structure}.

    \begin{figure}[ht]
      \centering
      \begin{minipage}{0.4\textwidth}
        \centering
        $$
        \begin{pmatrix}
          0 & \bullet & 0 & \bullet & \bullet & 0 \\
          0 & \bullet & 0 & 0 & \bullet & 0 \\
          0 & 0 & \bullet & 0 & \bullet & \bullet \\
        \end{pmatrix}
        $$
      \end{minipage}
      \begin{minipage}{0.4\textwidth}
        \centering
        $$
        \begin{matrix}
          JP & : & 1 & 1 & 2 &   &   \\
           J & : & 0 & 2 & 3 & 0 & 3 \\
          JS & : & 0 & 3 & 0 &   &   \\
        \end{matrix}
        $$
      \end{minipage}
      \caption[Matrix with corresponding C3SR representation (structural information only).]{\textbf{Matrix with
      corresponding C3SR representation (structural information only).} The non-zeros are denoted as black dots. The
      $0^{\text{th}}$ and $2^{\text{th}}$ row exhibit the same column index pattern $\big(0,2,3\big)$ at different peg
      index positions of $1$ and $2$, respectively. Thus the $0^{\text{th}}$ and the $2^{\text{th}}$ element of JS
      point J[0], which is the first element of this unique pattern. The $1^{\text{th}}$ row has a unique pattern which
      is stored after the previous pattern within J at offset $3$ which JS[1] accordingly points.}
      \label{fig:c3sr_example_structure}
    \end{figure}

    The matrix's non-zeros' numerical values are represented using two dense arrays: V, which contains the actual values
    and VS, the index-pointer into V which relates to V in the same way JS relates to J. As with J, row values are
    stored uniquely: Multiple rows with identical non-zeros (with possibly differing columns) lead to a single set of
    entries in V.

    \begin{description}[align = left, labelwidth = 4cm]
      \item [V - \emph{Values}] \hfill \\
        Non-zeros' numerical values. Duplicates across rows with the same pattern are stored only once.
      \item [VS - \emph{Values' index-pointers}] \hfill \\
        Index-pointer to each row's first element within V. One element per row in matrix.
    \end{description}

    An additional sixth array RS stores the number of non-zeros in each row of the matrix. This information is required
    as the index-pointer arrays JS and VS point a row's first element within J and V but, in contrast to the general CSR
    format, does not contain the information about how long the segment pertaining to the row within those arrays is. By
    virtue of the patterns' and values' storage schemes multiple rows' non-zeros' patterns and values may correspond to
    the very same portion of J and V such that the end of that portion cannot be retrieved from the corresponding
    index-pointer arrays JS and VS as could be done for the CSR format by taking the difference of two consecutive
    index-pointers from RP.

    \begin{description}[align = left, labelwidth = 4cm]
      \item [RS - \emph{Row sizes}] \hfill \\
        Number of non-zeros for each row. One element per row in matrix.
    \end{description}

    Another exemplary matrix and its full C3SR format representation are given in Figure \ref{fig:c3sr_example_full}.

    \begin{figure}[ht]
      \centering
      \begin{minipage}{0.4\textwidth}
        \centering
        $$
        \begin{pmatrix}
          1 & 2 & 0 & 3 & 0 & 0 \\
          0 & 1 & 2 & 0 & 3 & 0 \\
          0 & 0 & 4 & 5 & 0 & 0 \\
          0 & 0 & 6 & 7 & 0 & 0 \\
        \end{pmatrix}
        $$
      \end{minipage}
      \begin{minipage}{0.4\textwidth}
        \centering
        $$
        \begin{matrix}
          JP & : & 0 & 1 & 2 & 2 &   &   &   \\
           J & : & 0 & 1 & 3 & 0 & 1 &   &   \\
          JS & : & 0 & 0 & 3 & 3 &   &   &   \\
           V & : & 1 & 2 & 3 & 4 & 5 & 6 & 7 \\
          VS & : & 0 & 0 & 3 & 5 &   &   &   \\
          \RS & : & 3 & 3 & 2 & 2 &   &   &   \\
        \end{matrix}
        $$
      \end{minipage}
      \caption[Matrix with corresponding full C3SR representation.]{\textbf{Matrix with corresponding full C3SR representation.} The $0^{\text{th}}$ and $2^{\text{th}}$ rows exhibit the same pattern and the same values, thus the values are stored only once. The last two rows also share a common pattern but with different values.}
      \label{fig:c3sr_example_full}
    \end{figure}

    By design of the storage scheme, sparse banded matrices such as the example shown in Figure
    \ref{fig:laplacian-example} require very few elements within J as a structured grid's inner nodes, whose
    corresponding rows in the sparse banded matrix exhibit the same pattern which is only stored once, comprise the vast
    majority of all nodes. Depending on the underlying physical problem this property also applies to V if all rows of
    an equal pattern share the same set of values. 

    As will be detailed in the following section \ref{subsec:matrix-vector-multiplication-schemes} a matrix-vector
    multiplication involves copiously accessing J and V the C3SR format greatly improves the spatial locality of the
    data fields most pertinent to arithmetic. A best-case example is given in Figure \ref{fig:c3sr-example-best-case}.

    \begin{figure}[ht]
      \centering
      \begin{minipage}{0.5\textwidth}
        \centering
        $$
        \begin{pmatrix}
          1 & 0 & 2 & 0 & 3 & 4 & 0 & 0 & 0 & 0 \\
          0 & 1 & 0 & 2 & 0 & 3 & 4 & 0 & 0 & 0 \\
            &   & \ddots &   & \ddots &   & \ddots & \ddots \\
          0 & 0 & 0 & 1 & 0 & 2 & 0 & 3 & 4 & 0 \\
          0 & 0 & 0 & 0 & 1 & 0 & 2 & 0 & 3 & 4 \\
        \end{pmatrix}
        $$
      \end{minipage}
      \begin{minipage}{0.4\textwidth}
        \centering
        $$
        \begin{matrix}
          JP & : & 0 & 1 & 2 & 3 & \cdots & N-1 \\
           J & : & 0 & 2 & 4 & 5 &        &     \\
          JS & : & 0 & 0 & 0 & 0 & \cdots &  0  \\
           V & : & 1 & 2 & 3 & 4 &        &     \\
          VS & : & 0 & 0 & 0 & 0 & \cdots &  0  \\
          \RS & : & 4 & 4 & 4 & 4 & \cdots &  4  \\
        \end{matrix}
        $$
      \end{minipage}
      \caption[Best-case sparse matrix with corresponding C3SR representation.]{\textbf{Best-case sparse matrix with corresponding C3SR representation.} The matrix exhibits a single pattern and all of its rows have the same values leading to minimally sized J and V optimal for the repeated accesses required for matrix-vector multiplication.}
      \label{fig:c3sr-example-best-case}
    \end{figure}

    Note that the structure of a sparse banded matrix derived from a fully structured grid is exclusively diagonal or
    locally uniform, i.e. the non-zeros either align into multiple diagonals as in Figure \ref{fig:laplacian-example} or
    contain blocks of square matrices when solving for multiple coupled entities. Thus the concept of decomposing a
    row's non-zeros' column indices into its peg index and patterns is slightly redundant but is retained for the sake
    of flexibility when dealing with matrices associated with grids which contain structured segments.

    \subsection{Run-Length-Increment Encoding}

      The storage scheme introduced above may be further refined by run-length encoding the index-pointer arrays VS and
      JS aswell as JP and RS, respectively. Sparse banded matrices derived from large structured grids contain slices of
      fully diagonal structure at regular intervals corresponding to nodes with contiguous indices in the inner volume
      of the grid where the adjacency pattern is not perturbed by the grid's boundaries. The matrix in Figure
      \ref{fig:laplacian-example} exhibits these slices consisting of 3 contiguous rows ($x$-dimension of the grid is 5
      minus 2 outer nodes). The size of these slices grows linearly with the grid's extent in the direction of
      increasing node index.

      The pattern index-pointers in JS increase linearly

  \section{Matrix-Vector Multiplication Schemes} \label{subsec:matrix-vector-multiplication-schemes}

    Aside from the basic CSR-like matrix-vector multiplication scheme, the C3SR format's storage scheme allows for a
    vectorized arithmetic scheme for sparse banded matrices. This section details the different matrix-vector
    multiplication schemes whose performance is then gauged in the subsequent section.

    \subsection{Basic CSR-like Multiplication Scheme} \label{subsubsec:basic-csr-like-multiplication-scheme}

      Algorithmically, the basic row-by-column matrix-vector multiplication scheme of the C3SR format is similar to that of the CSR format. Differences arise only in the C3SR's additional offset $JP[k]$ which is applied to each relative column index prior to accessing the argument vector $x$ as shown in Figure \ref{fig:c3sr_matvecmult_basic}.

      \begin{figure}[ht]
        \centering
        $$
        \begin{matrix}
          \text{CSR}  & : & \sum \limits_{\alpha = 0}^{\RP[k+1] - \RP[k] - 1} & \bigg( & V[\RP[k] + \alpha]   & \cdot & x\big[\CI[\RP[k] + \alpha]\big] & \bigg)\\
          \vspace{0.3cm} \\
          \text{C3SR} & : & \sum \limits_{\alpha = 0}^{\RS[k] - 1} & \bigg( & V\big[VS[k] + \alpha\big] & \cdot & x\bigg[J\big[JS[k] + \alpha\big] + JP[k]\bigg] & \bigg) \\
        \end{matrix}
        $$
        \caption[Matrix-vector multiplication schemes of the C3SR and CSR format in comparison.]{\textbf{Matrix-vector multiplication schemes of the C3SR and CSR format in comparison.} The $k^{\text{th}}$ component of the product $Ax$ is shown. The C3SR format accesses V and J from their dedicated index-pointer arrays while the CSR format utilizes one common row-pointer array \RP. Additionally, the C3SR format requires the relative column indices to be offset by the peg column index for each row.}
        \label{fig:c3sr_matvecmult_basic}
      \end{figure}

      As the C3SR format utilizes additional arrays to store index-pointers the number of memory accesses increases likewise. Assuming a general case of a large sparse matrix devoid of any regularity in its structure whose size exceeds the machine's cache capacity the CSR format's arithmetic performance will be better proportional to the difference in the storage size as the above schemes yield memory bound computations involving few trivial arithmetic operations.

      On the flipside, as demonstrated in the previous section sparse banded matrices facilitate very compact storage allowing a small segment of memory, corresponding to possibly only a few cache lines, to contain the matrix's complete structural information and, depending on the underlying physical problem, even the numeric values. This can drastically improve the arithmetic performance of the C3SR format despite the more complex memory access scheme as will be shown in the benchmark section.

    \subsection{Vectorized SIMD Multiplication Scheme} \label{subsubsec:vectorized-simd-multiplication-scheme}

      In addition to the performance gain utilizing the basic CSR-like multiplication scheme due to better data locality, the data layout of a structured matrix in the C3SR format allows for the matrix-vector multiplication to be implemented utilizing SIMD parallelism. The matrix's data is laid out in memory in such a way that the matrix-vector multiplication may be performed for multiple rows at a time using vectorization. Depending on the composition of the matrix corresponding to the types of problems mentioned in the introduction (see \ref{subsec:structured-grid-matrices}) three different but very similar multiplication schemes can be devised.

      \paragraph{SIMD Scheme I: Diagonal structure}

      Suppose that a horizontal slice of rows $r, \ldots, r+k$ has a diagonal structure with the diagonals starting at indices $s, t, \ldots $, such as is depicted in Figure \ref{fig:simd_scheme_diag}. The matrix-vector multiplication for each of the slice's rows is composed of the same number of summands, one per diagonal. Each of the summands is a product of the diagonal's non-zero entry and the argument's corresponding element, starting out with $a_{r,s} \cdot x_s$ for the first diagonal's initial element and $a_{r,t} \cdot x_t$ for the second diagonal. For each subsequent row the matrix element's indices are incremented as well as the argument vector's index, commencing with $a_{r+1, s+1} \cdot x_{s+1}$ for the first diagonal's second element and accordingly for each other diagonal.

      As each diagonal's summand accesses the argument vector's elements in consecutive fashion, for example, $x_s, x_{s+1}, \ldots, x_{s + k}$ for the first diagonal, the access to the argument vector's elements can be vectorized. The other summand's constituents, the diagonal's elements, are stored within V at a fixed stride which is equal to the number of diagonals. This is due to the fact that the matrix slice's non-zero elements are stored row-by-row while each row has the same number of elements. Thus, for the general case when the rows' values don't match, $a_{r, s}$ is located as many elements before $a_{r+1, s+1}$ within V as there are diagonals.

      In SIMD terms a stride-gather and a load are required for the diagonal's elements and the argument vector, respectively. The two vector registers are then multiplied and the result is then added onto the next product pertaining to the subsequent diagonal. After the final vectorized addition the result is written into the output buffer using a vectorized store.

      \begin{figure}[ht]
        \centering
        $$
        \begin{pmatrix}
          \\
          \cdots & 0 & a_{r,s} & 0 & \cdots & a_{r,t} & 0 & \cdots & \cdots & \cdots \\
          \cdots & \cdots & 0 & a_{r+1,s+1} & 0 & \cdots & a_{r+1,t+1} & 0 & \cdots & \cdots \\
          \cdots & \cdots & \cdots & 0 & a_{r+2,s+2} & 0 & \cdots & a_{r+2,t+2} & 0 & \cdots \\
          \\
        \end{pmatrix}
        $$
        $$
        \begin{matrix}
          \begin{bmatrix}
            a_{r,s}     \\
            a_{r+1,s+1} \\
               \vdots   \\
            a_{r+k,s+k} \\
          \end{bmatrix} & \cdot & \begin{bmatrix}
                                    x_s      \\
                                    x_{s+1}  \\
                                      \vdots \\
                                    x_{s+k}  \\
                                  \end{bmatrix} & + & \begin{bmatrix}
                                                      a_{r,t}     \\
                                                      a_{r+1,t+1} \\
                                                        \vdots    \\
                                                      a_{r+k,t+k} \\
                                                      \end{bmatrix} & \cdot & \begin{bmatrix}
                                                                                x_t \\
                                                                                x_{t+1} \\
                                                                                \vdots \\
                                                                                x_{t+k}
                                                                              \end{bmatrix} & + & \cdots & = \begin{bmatrix}
                                                                                                                 y_{r} \\
                                                                                                                 y_{r+1} \\
                                                                                                                 \vdots \\
                                                                                                                 y_{r+k}
                                                                                                                \end{bmatrix}\\

        \end{matrix}
        $$
        \caption[Matrix slice with diagonal structure and corresponding vectorized matrix-vector multiplication scheme.]{\textbf{Matrix slice with diagonal structure (above) and corresponding vectorized matrix-vector-multiplication scheme $\bm{Ax = y}$ (below).} A diagonal's elements are located within V at a fixed stride equal to the number of diagonals. Depending on the hardware capabilities this may make the gather-load operation more performant.}
        \label{fig:simd_scheme_diag}
      \end{figure}

      \paragraph{SIMD Scheme II: Diagonal structure and identical values}

      For the case of a matrix slice whose structure is diagonal and whose rows share the same values, which equates to all values within a diagonal being identical, e.g. $a_{r,s} = a_{r+1, s+1} = \ldots$ for the first diagonal, the previous multiplication scheme is simplified in that the argument vector's values contained in the vector register are simply scaled by the diagonal's value instead of being subjected to a vectorized multiplication (Figure \ref{fig:simd_scheme_diag_collated}). Thus, in contrast to the previous multiplication scheme no SIMD gather is required to compute a matrix-vector product which significantly reduces the operation's comprexity resulting in faster execution times.

      \begin{figure}[ht]
        \centering
        $$
        \begin{pmatrix}
          \\
          \cdots & 0 & a_{r,s} &  0 & \cdots & a_{r,t} & 0 & \cdots & \cdots & \cdots \\
          \cdots & \cdots & 0 & a_{r+1,s+1} & 0 & \cdots & a_{r+1,t+1} & 0 & \cdots & \cdots \\
          \cdots & \cdots & \cdots & 0 & a_{r+2,s+2} & 0 & \cdots & a_{r+2,t+2} & 0 & \cdots \\
          \\
        \end{pmatrix}
        $$
        %$$
        %V: \begin{pmatrix}
        %  \ldots & a_{r,s} & a_{r,t} & \ldots & \ldots \\
        %\end{pmatrix}
        %$$
        $$
        \begin{matrix}
          a_{r,s} & \cdot & \begin{bmatrix}
                                    x_s      \\
                                    x_{s+1}  \\
                                      \vdots \\
                                    x_{s+k}  \\
          \end{bmatrix} & + & a_{r,t} & \cdot & \begin{bmatrix}
                                                x_t \\
                                                x_{t+1} \\
                                                \vdots \\
                                                x_{t+k}
                                                                              \end{bmatrix} & + & \cdots & =  \begin{bmatrix}
                                                                                                                 y_{r} \\
                                                                                                                 y_{r+1} \\
                                                                                                                 \vdots \\
                                                                                                                 y_{r+k}
                                                                                                                \end{bmatrix}
        \end{matrix}
        $$
        \caption[Matrix slice with diagonal structure and uniform values and corresponding vectorized matrix-vector-multiplication scheme.]{\textbf{Matrix slice with diagonal structure and uniform values (above) and corresponding vectorized matrix-vector-multiplication scheme $\bm{Ax = y}$ (below).} All values within a diagonal are equal simplifying the arithmetic scheme by removing a stride-load operation and a vectorized multiplication in favor of a simple scaling of the argument vector's vector register by the diagonal's value.}
        \label{fig:simd_scheme_diag_collated}
      \end{figure}

      \paragraph{SIMD Scheme III: Uniform structure}

      The third practically relevant case which arises from solving PDEs on multiple coupled entities is matrix slices whose structure is uniform, i.e. each row's non-zero elements are located within the same columns in the matrix slice. This leads to the circumstance that the argument vector's element required for the partial sum involving a matrix column is constant for the whole column (see Figure \ref{fig:simd_scheme_uniform}). Thus the argument vector's elements serve as scaling constants for the vector register containing the matrix's elements, which are again spread at a fixed stride throughout V. Note that for sparse banded matrices derived from structured grids the values usually differ between rows as the matrix would be singular otherwise. Thus, in practice, the matrix-vector multiplication does not collapse to scalar multiplication.

      \begin{figure}[ht]
        \centering
        $$
        \begin{pmatrix}
          \\
          \cdots & 0 & a_{r,s} &  0 & \cdots & a_{r,t} & 0 & \cdots \\
          \cdots & 0 & a_{r+1,s} & 0 & \cdots & a_{r+1,t} & 0 & \cdots \\
          \cdots & 0 & a_{r+2,s} & 0 & \cdots & a_{r+2,t} & 0 & \cdots \\
          \\
        \end{pmatrix}
        $$
        %$$
        %V: \begin{pmatrix}
        %  \ldots & a_{r,s} & a_{r,t} & \ldots & a_{r+1,s} & a_{r+1,t} & \ldots & a_{r+2,s} & a_{r+2,t} & \ldots \\
        %\end{pmatrix}
        %$$
        $$
        \begin{matrix}
          \begin{bmatrix}
            a_{r,s}     \\
            a_{r+1,s} \\
               \vdots   \\
            a_{r+k,s} \\
          \end{bmatrix} & \cdot & x_s & + & \begin{bmatrix}
                                              a_{r,t}     \\
                                              a_{r+1,t} \\
                                                \vdots    \\
                                              a_{r+k,t} \\
                                            \end{bmatrix} & \cdot & x_t & + \cdots & = \begin{bmatrix}
                                                                                       y_{r}   \\
                                                                                       y_{r+1} \\
                                                                                       \vdots  \\
                                                                                       y_{r+k} \\
                                                                                     \end{bmatrix}
        \end{matrix}
        $$
        \caption[Matrix slice with uniform structure and corresponding vectorized matrix-vector-multiplication scheme.]{\textbf{Matrix slice with uniform structure (above) and corresponding vectorized matrix-vector-multiplication scheme $\bm{Ax=y}$ (below).} Each non-zero column's elements are located within V at a fixed stride offset corresponding to the number of non-zero columns. The argument vector's elements serve as scaling constants for the vector register containing the matrix columns' values.}
        \label{fig:simd_scheme_uniform}
      \end{figure}

      All of the arithmetic schemes described above can be extended to vector registers of arbitrary length and are thus theoretically able to scale with additional hardware capabilities. Of course, in order to utilize the above-mentioned computation schemes, a preliminary analysis of the matrix's structure has to be performed in order to determine the segments which can facilitate vectorized arithmetic. Due to the storage scheme of the C3SR format, this can be done very cheaply since if a matrix segment has a diagonal or uniform structure all of the rows exhibit the same pattern and hence the same values in JS. For uniformly structured segments all peg indices are identical, while for a diagonally structured segment the peg indices are consecutive integers. Additionally, if the rows' values are identical for the case of a diagonal structure the rows' index-pointers VS are also identical.


