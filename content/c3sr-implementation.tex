\chapter{Implementation}

  \section{Run-Length-Increment Encoding}

    \Todo{Inhalt}

    The storage scheme introduced above may be further refined by run-length encoding the index-pointer arrays VS and
    JS aswell as JP and RS, respectively. Sparse banded matrices derived from large structured grids contain slices of
    fully diagonal structure at regular intervals corresponding to nodes with contiguous indices in the inner volume
    of the grid where the adjacency pattern is not perturbed by the grid's boundaries. The matrix in Figure
    \ref{fig:laplacian-example} exhibits these slices consisting of 3 contiguous rows ($x$-dimension of the grid is 5
    minus 2 outer nodes). The size of these slices grows linearly with the grid's extent in the direction of
    increasing node index.

    The pattern index-pointers in JS increase linearly ...

  \section{Matrix-Vector Multiplication}

    \subsection{Parallelization}

      As opposed to the CSR-format spreading the arithmetic workload of matrix-vector multiplications evenly among
      software threads is less trivial. The CSR-format's row-pointer array contains the scan of the number of non-zeros
      up to a row's respective index and as the number of non-zeros in a row is roughly proportional to its arithmetic
      workload ($n_\text{nnz}$ multiplications and $(n_\text{nnz} - 1)$ additions per $n_\text{nnz}$ non-zeros)
      spreading the workload evenly among $n_T$ theads is trivial requiring the row-pointer array to be partitioned such
      that the partitions' values span evenly sized ranges of indices. As the row-pointer array is sorted in ascending
      fashion this can be achieved very efficiently by $(n_T - 1)$ binary searches.
