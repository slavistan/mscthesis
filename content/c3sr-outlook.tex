\chapter{Conclusion and Outlook}

  The C3SR-format has been evalutated in terms of its compression scheme and its arithmetic performance based on two
  families of matrices derived from structured grids. The compression scheme is found to perform well in terms of
  compression ratio and storage size per nonzero in the matrix even when applied to matrices which deviate from the
  optimal case of fully structured grids.

  In contrast, matrix-vector multiplication achieved only mediocre results on three different machines. Only in one of
  six configurations was the arithmetic performance shown to be limited by system bandwidth. Vectorization was shown to
  provide significant benefits or, at worst, be equivalent to a scalar implementations. For a subset of the systems
  used to gauge arithmetic performance the chosen problem's domain sizes proved to be too small to render the thread
  sychronization overhead irrelevant. Introduction of perturbations to the fully regular problems showed a significant
  loss of performance, partly due to impairment of the ability to vectorize computations and because of deterioration in
  the predictability of memory accesses and data locality.

  Relevant future work may include larger grid sizes in order to hide synchronization overhead. Similar to the matrices
  studied in this thesis, sparse banded matrices derived from problems based on higher dimensional entities such as
  vectors or matrices consist of segments of arbitrary or collated composition of values and uniform structure and,
  based on the premises of the compression scheme, should compresse equally well. Their arithmetic performance remains
  to be evaluated. Introspection into the distribution of memory accesses into the objects' data is expected to
  provide insight to the stark drop in performance of arithmetic with perturbed matrices. Likewise, a CPU profiler will
  aid in finding performance bottlenecks where the system's memory bandwidth is not exhausted.
