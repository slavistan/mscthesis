\chapter{Outlook}

Scan of row-sizes for load-balancing mvm.

Further shrink COMP by including it indirectly into JS. COMP analysis is much faster on the RLI-enc'd arrays.

Further control over NUMA allocation. Mapping of arrays' sections to NUMA nodes can only be done perfectly AFTER the
object has been created and it known, which NUMA nodes access which sections of V (complex matrices).

\Todo{SIMD kann nicht sinnvoll genutzt werden, höchstens für kleine Objekte. Skalar MVM genauso schnell, bietet aber
potential für Zukunft und Systeme mit großen Caches -> KNL}

\Todo{Proper NUMA control - Tbb does not allow for it; hacky}

\Todo{Note to RLE: Falls für höherdimensionalen Entitäten gelöst wird, entstehen kleine Blockmatrizen-Einträge, die
Objekt total aufblähen, falls RLE verwendet wird}.

%% Stuff from SRP

  This report introduces the threefold compressed sparse row matrix storage format (C3SR) which has been designed to
  adapt the CSR format to improve the performance of matrix-vector multiplication for sparse banded matrices derived
  from structured grids. Significant performance benefits have been demonstrated for two different sparse banded
  matrices based on a $100 \times 100 \times 100$ grid utilizing the basic CSR-like matrix-vector multiplication scheme
  of the C3SR format.

  A vectorized arithmetic scheme could only be shown to provide performance improvements for the matrix whose storage
  size in memory is significantly smaller implying that the matrix-vector multiplication is bound by the system's memory
  bandwidth. Thus, for the machines utilized for this work, the applicability of vectorization for the arithmetic
  depends strongly on the sparse banded matrix in question. For some systems, it showed worse performance than the
  baseline CSR-like implementation.

  \section*{Future Work}

    In order to gauge arithmetic performance, this project uses few synthetic matrices which, albeit being realistic
    examples, cover only very little of the total spectrum of sparse banded matrices that occur in real-life
    applications. Especially heterogeneous domains as discussed in section \ref{subsec:structured-grid-matrices} yield
    sparse matrices which are only partially structured. These scenarios have to be investigated separately in terms of
    arithmetic performance.
